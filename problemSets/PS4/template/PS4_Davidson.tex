\documentclass[12pt,letterpaper]{article}
\usepackage{graphicx,textcomp}
\usepackage{natbib}
\usepackage{setspace}
\usepackage{fullpage}
\usepackage{color}
\usepackage[reqno]{amsmath}
\usepackage{amsthm}
\usepackage{fancyvrb}
\usepackage{amssymb,enumerate}
\usepackage[all]{xy}
\usepackage{endnotes}
\usepackage{lscape}
\newtheorem{com}{Comment}
\usepackage{float}
\usepackage{hyperref}
\newtheorem{lem} {Lemma}
\newtheorem{prop}{Proposition}
\newtheorem{thm}{Theorem}
\newtheorem{defn}{Definition}
\newtheorem{cor}{Corollary}
\newtheorem{obs}{Observation}
\usepackage[compact]{titlesec}
\usepackage{dcolumn}
\usepackage{tikz}
\usetikzlibrary{arrows}
\usepackage{multirow}
\usepackage{xcolor}
\newcolumntype{.}{D{.}{.}{-1}}
\newcolumntype{d}[1]{D{.}{.}{#1}}
\definecolor{light-gray}{gray}{0.65}
\usepackage{url}
\usepackage{listings}
\usepackage{color}

\definecolor{codegreen}{rgb}{0,0.6,0}
\definecolor{codegray}{rgb}{0.5,0.5,0.5}
\definecolor{codepurple}{rgb}{0.58,0,0.82}
\definecolor{backcolour}{rgb}{0.95,0.95,0.92}

\lstdefinestyle{mystyle}{
	backgroundcolor=\color{backcolour},   
	commentstyle=\color{codegreen},
	keywordstyle=\color{magenta},
	numberstyle=\tiny\color{codegray},
	stringstyle=\color{codepurple},
	basicstyle=\footnotesize,
	breakatwhitespace=false,         
	breaklines=true,                 
	captionpos=b,                    
	keepspaces=true,                 
	numbers=left,                    
	numbersep=5pt,                  
	showspaces=false,                
	showstringspaces=false,
	showtabs=false,                  
	tabsize=2
}
\lstset{style=mystyle}
\newcommand{\Sref}[1]{Section~\ref{#1}}
\newtheorem{hyp}{Hypothesis}

\title{Problem Set 4}
\date{Due: April 4, 2022}
\author{Applied Stats II}


\begin{document}
	\maketitle
	\section*{Instructions}
	\begin{itemize}
		\item Please show your work! You may lose points by simply writing in the answer. If the problem requires you to execute commands in \texttt{R}, please include the code you used to get your answers. Please also include the \texttt{.R} file that contains your code. If you are not sure if work needs to be shown for a particular problem, please ask.
		\item Your homework should be submitted electronically on GitHub in \texttt{.pdf} form.
		\item This problem set is due before class on Monday April 4, 2022. No late assignments will be accepted.
		\item Total available points for this homework is 80.
	\end{itemize}

	\vspace{.25cm}
\section*{Question 1}
\vspace{.25cm}
\noindent We're interested in modeling the historical causes of infant mortality. We have data from 5641 first-born in seven Swedish parishes 1820-1895. Using the "infants" dataset in the \texttt{eha} library, fit a Cox Proportional Hazard model using mother's age and infant's gender as covariates. Present and interpret the output.

\begin{lstlisting}[language=R]
	
	#Getting started:
	
	# remove objects
	rm(list=ls())
	# detach all libraries
	detachAllPackages <- function() {
		basic.packages <- c("package:stats", "package:graphics", "package:grDevices", "package:utils", "package:datasets", "package:methods", "package:base")
		package.list <- search()[ifelse(unlist(gregexpr("package:", search()))==1, TRUE, FALSE)]
		package.list <- setdiff(package.list, basic.packages)
		if (length(package.list)>0)  for (package in package.list) detach(package,  character.only=TRUE)
	}
	detachAllPackages()
	
	# load libraries
	pkgTest <- function(pkg){
		new.pkg <- pkg[!(pkg %in% installed.packages()[,  "Package"])]
		if (length(new.pkg)) 
		install.packages(new.pkg,  dependencies = TRUE)
		sapply(pkg,  require,  character.only = TRUE)
	}
	
	lapply(c("survival", "eha", "tidyverse", "ggfortify", "stargazer"),  pkgTest)
	
	
	# And so it begins:
	
	data(infants)
	
	imr <- with(infants, Surv(enter, exit, event))
	
	cox <- coxph(imr ~ sex + age, data = infants)
	summary(cox)
	drop1(cox, test = "Chisq")
	stargazer(cox, type = "text")
	
	
	cox_fit <- survfit(cox)
	autoplot(cox_fit)
	
	newdat <- with(infants, 
	data.frame(
	sex = c("male", "female"), age="age"
	)
	)
	
	autoplot(newdat)
	
	plot(survfit(cox, newdata = newdat), xscale = 12,
	conf.int = T,
	ylim = c(0.6, 1),
	col = c("red", "blue"),
	xlab = "age",
	ylab = "Survival proportion",
	main = "Survival")
	legend("bottomleft",
	legend=c("Male", "Female"),
	lty = 1, 
	col = c("red", "blue"),
	text.col = c("red", "blue"))
	
	# Adding an interaction
	cox.int <- coxph(imr ~ sex * age, data = infants)
	summary(cox.int)
	drop1(cox.int, test = "Chisq")
	stargazer(cox.int, type = "text")
	
	autoplot(cox.int)
	
	coxfit <- survfit(cox.int)
	
	autoplot(coxfit)
	
	
	## I am trying to plot this to get a better understanding of the data, but I am having no luck. 
	# It seems, however, that the older a person is there would by a higher rate of death. But, again,
	# I am having a hard time figuring this out.

\end{lstlisting}

\end{document}
